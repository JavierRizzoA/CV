%!TEX TS-program = xelatex
\documentclass[]{friggeri-cv}
\usepackage{afterpage}
\usepackage{hyperref}
\usepackage{color}
\usepackage{xcolor}
\usepackage{fontspec}
\hypersetup{
    pdftitle={Javier Rizzo CV},
    pdfauthor={Javier Rizzo},
    pdfsubject={},
    pdfkeywords={},
    colorlinks=false,       % no lik border color
   allbordercolors=white    % white border color for all
}
\addbibresource{bibliography.bib}
\RequirePackage{xcolor}
\definecolor{pblue}{HTML}{0395DE}

\newfontface\lserif{Liberation Serif}
\newcommand{\Csh}{C{\lserif\#}}

\begin{document}
\header{Javier }{Rizzo}
      {Cybernetics Electronics Engineering Student}
      
% Fake text to add separator      
\fcolorbox{white}{gray}{\parbox{\dimexpr\textwidth-2\fboxsep-2\fboxrule}{%
...
}}

% In the aside, each new line forces a line break
\begin{aside}
    ~
    ~
    ~
  \section{Address}
    Av. La Rivera, 1724, 
    El Lienzo
    Mexicali, B.C., México.
    C.P. 21258
    ~
  \section{Phone}
    +52 (686) 288 73 57
    ~
  \section{Email}
    \href{mailto:javierrizzoa@gmail.com}{\textbf{javierrizzoa@}gmail.com}
    ~
  \section{Web \& Git}
    \href{http://javierrizzo.com}{javierrizzo.com}
    \href{https://github.com/javierrizzoa}{github.com/javierrizzoa}
    ~
  \section{Programming}
  \textbf{Fluent in }
  JavaScript, Python, \Csh{}, Java, Haxe
  \textbf{Also know }
  C, C++, Ruby, VB.Net
    ~
  \section{Other Skills}
  HTML5, CSS3, Git, Node.js, SQL, WPF, Electronics, Arduino, \LaTeX
  ~
  \section{Hobbies}
  Writing, Videogames, Competitive programming, Linux and VIM enthusiast.
  ~
  \section{Languages}
    \textbf{Spanish}\includegraphics[scale=0.40]{img/5stars.png}
    \textbf{English}\includegraphics[scale=0.40]{img/4stars.png}
\end{aside}

\section{Education}
  \entry
    {2014 - now}
    {Bachelor of Engineering in Cybernetics Electronics}
    {Centro de Enseñanza Técnica y Superior (CETYS)}
    {Expected to graduate in July 2018}

\section{Side Projects}
Available on my GitHub (\href{https://github.com/javierrizzoa}{github.com/javierrizzoa})

\section{Experience}
    \entry
    {2016 - now}
    {Programming contractor}
    {Skyworks Inc (Mexicali, México)}
    {Sofware development with \Csh{} and MSSQL}
    \entry
    {2016 - now}
    {IT Intern}
    {CETYS University (Mexicali, México)}
    {Database management, MS Access programmer}
    \entry
    {2015}
    {Videogame Programming Workshop Instructor}
    {Trascendencias, CETYS University (Mexicali, México)}
    {Workshop instructor of videogame programming basics (in Haxe).}
    \entry
    {2015}
    {We Can Code Hackathon}
    {We Can Code Hackathon (Ensenada, México)}
    {24-hour hackathon were we built "Edumotion" using JavaScript and a LeapMotion.}
    \entry
    {2015 - now}
    {Instructor at El Garage Project Hub}
    {El Garage, Project Hub (Mexicali, México)}
    {Instructor of programming and electronics workshops at the first makerspace on the Mexican Northwest.}
    \entry
    {2015 - 2016}
    {SentinelFox Engineering Mentor}
    {SentinelFox Engineering (Mexicali, México}
    {Mentor of the FIRST Robotics team of CETYS Highschool.}
    \entry
    {2014}
    {We Can Code Hackathon}
    {We Can Code Hackathon (Ensenada, México)}
    {8-hour hackathon where we built "Whiteboard", using \Csh{} and a Wiimote.}
    \entry
    {2014}
    {Android Development Workshop Instructor}
    {Markë, CETYS University (Mexicali, México)}
    {Workshop of the basics of Android development.}
    \entry
    {2012 - 2014}
    {Programming and Control Leader}
    {Solar Engineering (Mexicali, México)}
    {Made the electric system and Java programming (with the FRC API) of two robots for the FIRST Robotics Competition 2013 and 2014.}

\section{Awards}
  \entry
    {2015}
    {We Can Code Hackathon - 2nd Place}
    {Major League Hacking (Ensenada, México)}
    {\emph{24-hour hackathon were we built "Edumotion" using JavaScript and a LeapMotion.}}
  \entry
    {2013 - 2015}
    {Concurso de Programación Proyecto Ingenería - 1st Place}
    {CETYS (Mexicali, México)}
    {\emph{College programming competition. Winner of the first place for three consecutive years (2013, 2014, and 2015.)}}
  \entry
    {2013}
    {Rookie All-Star Award}
    {FIRST Robotics Competition (San Diego, CA)}
    {\emph{Robotics Competition, where the challenge was to make a robot that shoots freesbies to some goals.}}
  \entry
    {2012}
    {Concurso de Programación Proyecto Ingenería - 2nd Place}
    {CETYS (Mexicali, México)}
    {\emph{College programming competition.}}
  \entry
    {2011}
    {Scenery Beta PC - 3rd Place}
    {SB IT Media, S.L. (Madrid, Spain)}
    {\emph{PC Software competition where I developed ROMFinder, a webscrapper using VB.Net}}
\\
\begin{flushleft}
\emph{Last update: October 1st, 2015}
\end{flushleft}
\begin{flushright}
\emph{Javier Rizzo}
\end{flushright}

\end{document}
