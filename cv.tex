\documentclass[]{friggeri-cv}
\usepackage{afterpage}
\usepackage{hyperref}
\usepackage{color}
\usepackage{xcolor}
\usepackage{fontspec}
\usepackage[spanish,english]{babel}
\hypersetup{
    pdftitle={Javier Rizzo CV},
    pdfauthor={Javier Rizzo},
    pdfsubject={},
    pdfkeywords={},
    colorlinks=false,       % no link border color
   allbordercolors=white    % white border color for all
}
\addbibresource{bibliography.bib}
\RequirePackage{xcolor}
\definecolor{pblue}{HTML}{0395DE}

\newcommand{\Csh}{C{\#}}

\newcommand{\langes}[1]{%
  \ifes\selectlanguage{english}#1\fi}
\newcommand{\langen}[1]{%
  \ifen\selectlanguage{english}#1\fi}

\begin{document}

%For some reason, I can't use langes/langen commands here...
\ifes
  \header{Javier }{Rizzo}
    {Ingeniero en Cibernética Electrónica}
\else
  \header{Javier }{Rizzo}
    {Electronic Cybernetics Engineer}
\fi
      
% Fake text to add separator      
\fcolorbox{white}{gray}{\parbox{\dimexpr\textwidth-2\fboxsep-2\fboxrule}{%
...
}}

% In the aside, each new line forces a line break
\begin{aside}
    ~
    ~
    ~
  \section{%
    \langes{Dirección}%
    \langen{Address}}
    Av. La Rivera, 1724, 
    El Lienzo
    Mexicali, B.C., México.
    C.P. 21258
    ~
  \section{%
    \langes{Teléfono}%
    \langen{Phone}}
    +52 (686) 288 73 57
    ~
  \section{Email}
    \href{mailto:javierrizzoa@gmail.com}{\textbf{javierrizzoa@}gmail.com}
    ~
  \section{Web \& Git}
    \href{http://javierrizzo.com}{javierrizzo.com}
    \href{https://github.com/javierrizzoa}{github.com/javierrizzoa}
    ~
    \section{%
      \langes{Programación}%
      \langen{Programming}}
  \textbf{%
    \langes{Fluído en}%
    \langen{Fluent in}}
  JavaScript, Python, \Csh{}, Java, Haxe
  \textbf{%
    \langes{Con experiencia en}%
    \langen{Also know}}
  C, C++, Ruby, VB.Net
    ~
  \section{%
    \langes{Otras habilidades}%
    \langen{Other Skills}}
    HTML5, CSS3, Android, iOS, Git, Node.js, SQL, WPF, \langes{Electrónica}\langen{Electronics}, Arduino, \LaTeX
  ~
  \section{Hobbies}
  \langes{Escribir, Videojuegos, Programación competitiva, Entusiasta de VIM y Linux.}%
  \langen{Writing, Videogames, Competitive programming, Linux and VIM enthusiast.}
  ~
  \section{%
    \langes{Idiomas}%
    \langen{Languages}}
    \textbf{%
      \langes{Español}%
      \langen{Spanish}}\includegraphics[scale=0.40]{img/5stars.png}
    \textbf{%
      \langes{Inglés}%
      \langen{English}}\includegraphics[scale=0.40]{img/4stars.png}
\end{aside}

%langes/langen commands couldn't be used here neither.
%I'm beginning to hate LaTeX's cryptic error messages.
%\section{\langes{Educación}\langen{Education}}
\ifes
  \section{Educación}
\else
  \section{Education}
\fi
  \entry
    {\langes{2014 - 2018}%
      \langen{2014 - 2018}}
    {\langes{Ingeniería en Cibernética Electrónica}%
      \langen{Bachelor of Engineering in Electronic Cybernetics}}
    {\langes{Centro de Enseñanza Técnica y Superior (CETYS)}%
      \langen{Centro de Enseñanza Técnica y Superior (CETYS)}}
    {\langes{Especializado en robótica y automatización industrial.}%
      \langen{Specialized in robotics and industrial automation.}}

\ifes
  \section{Proyectos personales}
\else
  \section{Side Projects}
\fi
\langes{Disponibles en mi GitHub (\href{https://github.com/javierrizzoa}{github.com/javierrizzoa})}%
\langen{Available on my GitHub (\href{https://github.com/javierrizzoa}{github.com/javierrizzoa})}

\ifes
  \section{Experiencia}
\else
  \section{Experience}
\fi
    \entry
    {\langes{2017}%
      \langen{2017}}
    {\langes{Interno de Ingeniería en Software}%
      \langen{Software Engineer Intern}}
    {\langes{Microsoft (Bellevue, Estados Unidos)}%
      \langen{Microsoft (Bellevue, USA)}}
    {\langes{Cortana @Work Calendar Engineering Team}%
      \langen{Cortana @Work Calendar Engineering Team}}
    \entry
    {\langes{2016}%
      \langen{2016}}
    {\langes{Programador contratista}%
      \langen{Programming contractor}}
    {\langes{Skyworks Inc (Mexicali, México)}%
      \langen{Skyworks Inc (Mexicali, Mexico)}}
    {\langes{Desarrollo de software con \Csh{} y MSSQL.}%
      \langen{Sofware development with \Csh{} and MSSQL.}}
    \entry
    {\langes{2016 - 2017}%
      \langen{2016 - 2017}}
    {\langes{Becario del área de informática}%
      \langen{IT Intern}}
    {\langes{CETYS Universidad (Mexicali, México)}%
      \langen{CETYS University (Mexicali, Mexico)}}
    {\langes{Administración de bases de datos, programador en MS Access.}%
      \langen{Database management, MS Access programmer.}}
    \entry
    {\langes{2015, 2016}%
      \langen{2015, 2016}}
    {\langes{Tallerista: Programación de videojuegos}%
      \langen{Videogame Programming Workshop Instructor}}
    {\langes{Trascendencias, CETYS Universidad (Mexicali, México)}%
      \langen{Trascendencias, CETYS University (Mexicali, Mexico)}}
    {\langes{Instructor del taller de "Programación de videojuegos" (en Haxe).}%
      \langen{Workshop instructor of videogame programming basics (in Haxe).}}
    \entry
    {\langes{2015 - ahora}%
      \langen{2015 - now}}
    {\langes{Maker en El Garage Project Hub}%
      \langen{Maker at El Garage Project Hub}}
    {\langes{El Garage, Project Hub (Mexicali, México)}%
      \langen{El Garage, Project Hub (Mexicali, Mexico)}}
    {\langes{Instructor en los talleres de programación y electrónica, desarrollo de proyectos y sysadmin en el primer makerspace del noroeste mexicano.}%
      \langen{Instructor of programming and electronics workshops, project developer, and sysadmin at the first makerspace on the Mexican Northwest.}}
    \entry
    {\langes{2012 - 2016}%
      \langen{2012 - 2016}}
    {\langes{Líder de programación y control}%
      \langen{FIRST Robotics Team Member}}
    {\langes{CETYS Universidad (Mexicali, México)}%
      \langen{CETYS University (Mexicali, Mexico)}}
    {\langes{Miembro del equipo de Robótica del 2012 al 2014, encargado del sistema eléctrico y la programación (en Java con el API de FRC) y mentor del equipo del 2015 al 2016.}%
      \langen{Robotics team member from 2012 to 2014, in charge of the electrical system and programming (Java with FRC's API), and mentor from 2015 to 2016.}}

\ifes
\section{Reconocimientos}
\else
\section{Awards}
\fi
  \entry
  {\langes{}%
    \langen{2015, 2017}}
  {\langes{We Can Code Hackathon - 2do lugar}%
    \langen{We Can Code Hackathon - 2nd Place}}
  {\langes{Major League Hacking (Ensenada, México)}%
    \langen{Major League Hacking (Ensenada, Mexico)}}
  {\emph{\langes{Hackathon de 24 horas, ganadores del 2do lugar en 2015 y 2017.}%
    \langen{24-hour hackathon, 2nd place winners in 2015 and 2017.}}}
  \entry
  {\langes{}%
    \langen{2013 - 2015}}
  {\langes{Concurso de Programación Proyecto Ingenería - 1er lugar}%
    \langen{Concurso de Programación Proyecto Ingenería - 1st Place}}
  {\langes{CETYS (Mexicali, México)}%
    \langen{CETYS (Mexicali, Mexico)}}
  {\emph{\langes{Concurso de programación universitario. Ganador del primer lugar por tres años consecutivos (2013, 2014 y 2015).}%
    \langen{College programming competition. Winner of the first place for three consecutive years (2013, 2014, and 2015).}}}
  \entry
  {\langes{2013}%
    \langen{2013}}
  {\langes{Rookie All-Star Award}%
    \langen{Rookie All-Star Award}}
  {\langes{FIRST Robotics Competition (San Diego, CA)}%
    \langen{FIRST Robotics Competition (San Diego, CA)}}
  {\emph{\langes{Competencia de robótica, donde el reto era hacer un robot que disparara freesbies a unas porterías.}%
    \langen{Robotics Competition, where the challenge was to make a robot that shoots freesbies to some goals.}}}
  \entry
  {\langes{2011}%
    \langen{2011}}
  {\langes{Scenery Beta PC - 3er lugar}%
    \langen{Scenery Beta PC - 3rd Place}}
  {\langes{SB IT Media, S.L. (Madrid, España)}%
    \langen{SB IT Media, S.L. (Madrid, Spain)}}
  {\emph{\langes{Concurso de programación donde creé ROMFinder, un webscrapper hecho en VB.Net}%
    \langen{PC Software competition where I developed ROMFinder, a webscrapper using VB.Net}}}
\\
\begin{flushleft}
\emph{\langes{Última actualización: Julio 8, 2018}%
  \langen{Last update: July 8th, 2018}}
\end{flushleft}
\begin{flushright}
\emph{Javier Rizzo}
\end{flushright}
\end{document}
